\section{Mathematical Analysis}
\label{sec:mathanalysis}

Before presenting experimental validation of our proposed MoRAM algorithm, we now perform a theoretical analysis of both the initialization and the descent stages, and provide an upper bound on the number of samples required for accurate sparse signal recovery under the Gaussian observation model.

\subsection{Analysis of initialization stage}
Recall that we perform the initialization in two steps:
(i) measurement correction step, and (ii) estimation step. We analyze them separately.

\subsubsection{Measurement correction step}\label{meascorr} In this step, based on the magnitude of the modulo measurements, we identify the measurements for which we can undo the effect of modulo transfer function by guessing the correct value of the bin-indices $\mathbf{p}$. We define the set $U$ as set of indices of all such measurements that can be corrected. %We determine the measurements to be corrected based on the method described in Section~\ref{sec:init}. 

In this analysis, our goal is to estimate the number of measurements ($N=\card(U)$) that can be corrected. Only these $N$ measurements would be used in the estimation step. Each element of $\mb{A}$ is chosen i.i.d. from a standard normal distribution. Therefore, $\mu_{\mb{A}_{ij}} = 0$ and $\sigma_{\mb{A}_{ij}}=1$. 

Recall that:
$$
y_{c,i} =\langle \mathbf{a_i} \cdot \mathbf{x^*} \rangle = \sum_{j=1}^{n}\mb{A}_{ij}x^*_{j}.
$$
Being a linear combination of Gaussian random variables, the corrected observations $y_{c,i}$ also follows Gaussian distribution:
$$
\mu_{y_{c}} = \sum_{j=1}^{n}x^{*}\mu_{\mb{A}_{ij}} = 0,
$$
$$
\sigma_{y_{c}} =\sum_{j=1}^{n}x^{*2}_{j} \sigma^2_{\mb{A}_{ij}} = \sum_{j=1}^{n}x^{*2}_{j} = \norm{\mathbf{{x}^*}}^2 = 1;
$$
$$
\implies y_{c,i} \sim \mathcal{N}\left(\mu= 0, \sigma = 1 \right)
$$
Hence, each element of $\mathbf{y_c}$ follows a zero mean Gaussian distribution with unit variance.

We can calculate $N$ using the fact that the compressive measurements $y_c$ follow the standard normal distribution, and the total number of measurements are $m$. We define $\mathcal{M}_{\alpha,\beta}$ as the set of all measurements lying in the interval $[\alpha,\beta]$. Thus,

Here, $\card\left(\mathcal{M}_{\alpha,\beta}\right)=mP(\alpha \leq y_{c,i}\leq \beta) = m\left(Q(\alpha)- Q(\beta)\right)$
where, $Q(\cdot)$ is the Gaussian Q-function defined as:
$$Q(t) = 1-\Phi(t)$$ 
with $\Phi(\cdot)$ being the CDF of standard normal distribution. We also note that:
 $$Q(-t) = 1 - Q(t).$$
The Q-function does not have a closed form expression. However, it can be bounded by the following functions where $\phi(\cdot)$ is standard normal density function:
$$
\left(\frac{t}{1+t^2}\right)\phi(t) < Q(t) < \frac{\phi(t)}{t}.
$$
Therefore, for $R > \rho$,
\begin{align}
N & = \card\left(\mathcal{M}_{-R + \rho,0}\right) + \card\left(\mathcal{M}_{0, R-\rho}\right) \nonumber \\
N &  = m\left(Q(-R+\rho)- Q(0)\right) + m\left(Q(0)- Q(R-\rho)\right) \nonumber \\
& = m\left(Q(-(R-\rho))- Q(R-\rho)\right) \nonumber \\
& = m\left(1-2Q(R-\rho)\right). \nonumber
\end{align}
Using the bounds above,
\begin{align}
N \geq m \left(1-2\frac{\phi(R-\rho)}{(R-\rho)} \right).
\label{eq:n_bound}
\end{align}
%\noindent case (ii). $R < \rho$: In this case,
%\begin{align}
%N & = \card\left(\mathcal{M}_{-\rho,-R}\right) + \card\left(\mathcal{M}_{R,\rho}\right) \nonumber \\
%N & = m\left(Q(-\rho)- Q(-R)\right) + m\left(Q(R)- Q(\rho)\right) \nonumber \\
%& = m\left(1 - Q(\rho)-1 +Q(R) +Q(R) -Q(\rho)\right) \nonumber \\
%& = 2m\left(Q(R)-Q(\rho)\right).
%\end{align}
%
%Using the bounds above,
%$$
%N \geq 2m \left(\frac{R}{(1+R^2)} \phi(R) - \frac{\phi(\rho)}{\rho}\right).
%$$
For a given \emph{total} number of observed measurements $m$, Eq.~\ref{eq:n_bound} provides the lower bound on the value of $N$, the number of \emph{correctable} measurements.

\subsubsection{Estimation step}
Using the corrected measurements (denoted as set $U$), initial estimate $\mb{x^0}$ is calculated as:
$$
\mb{x^0} = H_s\left(\frac{1}{N}\sum_{i=1}^{N}y^{init}_{c,U,i}a_{U,i}\right).
$$
where $H_s(\cdot)$ denotes the hard thresholding operator (that retains the $s$-largest magnitude coefficients of a given vector).  We use the versions of $\mb{y}$ and $\mb{A}$ truncated to the row indices belonging to set $U$, and column indices to set $S$. We denote such sub-matrices with the subscript $U\times S$. 

We can now prove that our initial estimate as constructed above is close to the true signal $\mb{x^*}$. We obtain:

\theorem{Let $\delta \in (0,1)$ and $\nu \geq 1$. The initial estimate $\mb{x^0}$ is the output of the algorithm~\ref{alg:RCM}. If the total number of corrected measurements satisfy, $m \geq C(\frac{\nu}{\kappa})^2s\left(1-2\frac{\phi(R-\rho)}{(R-\rho)} \right)^{-1}$, where $\kappa = \frac{\delta}{2}$, $\phi(\cdot)$ represents standard normal pdf; then  with probability at least $1 - 2\left(\frac{en}{s}\right)^s\exp\left(-\nu^2s\right)$ we have,
	$$
	\norm{\mb{x^0} - \mb{x^*}}_2 \leq \delta \norm{\mb{x^*}}_2.
	$$
	\label{th:1}
}

\proof {We define $\mb{\tilde{x}^0}$ as the value of the initial estimate before the hard thresholding step:
$$
\mb{\tilde{x}^0} = \frac{1}{N}\sum_{i=1}^{N}y^{init}_{c,U,i}a_{U,i} =  \frac{1}{N}\mb{A}_U^\t\mb{y_{c,\mathnormal{U}}}.
$$
Substituting $\mb{y_{c,\mathnormal{U}}}=\mb{A}_U\mb{x^*}$,
$$
\mb{\tilde{x}^0} = M = \frac{1}{N}\mb{A}_U^\t \mb{A}_U\mb{x^*}.
$$

We note that each row of our truncated Gaussian measurement matrix ($\mb{A}_{\scriptscriptstyle{U\times S}}$) is independent, and also follows the Gaussian distribution with zero mean.  
We denote this distribution with Gaussian random vector $Z$ in $\R^s$, and arrange $N$ rows as the independent samples from the distribution; $Z_i := \mb{A}_{\scriptscriptstyle{U\times S},i}$. 
Recall that the covariance matrix of $Z$ can be calculated as $\Sigma = \mathbb{E}Z\otimes Z$,
$$
\Sigma= \mathbb{E}Z\otimes Z = \mathbb{E} ZZ^\t = \diag{\mathbb{E}z_1^2,\mathbb{E}z_2^2,...,\mathbb{E}z_n^2} = I_n.
$$

Now, calculating the sample covariance matrix of $Z$ using the samples $Z_i = \mb{A}_{\scriptscriptstyle{U\times S},i}$,
$$
\Sigma_N = \frac{1}{N}\sum_{i=1}^{N}Z_i \otimes Z_i = \frac{1}{N} \mb{A}_{\scriptscriptstyle{U\times S}}^\t \mb{A}_{\scriptscriptstyle{U\times S}}.
$$

Given $N \geq C\left(\frac{\nu}{\kappa}\right)^2s$, we invoke Corollary 5.50 of~\cite{vershynin2010introduction} which relates $\Sigma_N$ and $\Sigma$ as follows, with probability at least $1 - 2\exp\left(-\nu^2s\right)$:

\begin{align*}
\norm{\Sigma_N - \Sigma}_2 &\leq \kappa,~~~~~\text{i.e.,} \\
\norm{\frac{1}{N} \mb{A}_{\scriptscriptstyle{U\times S}}^\t \mb{A}_{\scriptscriptstyle{U\times S}} - I_s}_2 &\leq \kappa \\
\end{align*}

We now use a standard covering number argument. Let us fix the $s$-sparse vector $\mb{x^*}$ within the unit norm ball. We can evaluate the operator norm in above equation over the set of $s-$sparse vectors in the unit norm ball.
\begin{align*}
\implies \sup_{\mb{x} \in S} \frac{\norm{\left(\frac{1}{N} \mb{A}_{\scriptscriptstyle{U\times S}}^\t \mb{A}_{\scriptscriptstyle{U\times S}} - I_n\right)\mb{x}}_2}{\norm{\mb{x}}_2} &\leq \kappa \\
\norm{\left(\frac{1}{N} \mb{A}_{\scriptscriptstyle{U\times S}}^\t \mb{A}_{\scriptscriptstyle{U\times S}} - I_n\right)\mb{x^*}}_2 &\leq \kappa \norm{\mb{x^*}_2} \\
 \norm{\left(\frac{1}{N} \mb{A}_{\scriptscriptstyle{U\times S}}^\t \mb{A}_{\scriptscriptstyle{U\times S}}\mb{x^*} - \mb{x^*}\right)}_2 &\leq \kappa \norm{\mb{x^*}_2} \\
 \norm{\mb{\tilde{x}^0} - \mb{x^*}}_2 &\leq \kappa \norm{\mb{x^*}}_2 
\end{align*}
with probability at least $1 - 2\exp\left(-\nu^2s\right)$ given the fixed $s-$sparse vector $\mb{x^*}$. Taking an union bound over $C \left(\frac{12}{\delta}\right)^s \binom{n}{s}$ such $s-$sparse vectors,
\begin{align*}
\mathbb{P}\left(\norm{\mb{\tilde{x}^0} - \mb{x^*}}_2 \leq \kappa \norm{\mb{x^*}}_2\right) & \geq 1 - 2 C \left(\frac{12}{\delta}\right)^s {n \choose s}\exp\left(-\nu^2s\right) \\ 
& \geq 1 - 2\left(\frac{Cn}{s}\right)^s\exp\left(-\nu^2s\right)
\end{align*}

The initial estimate $\mb{x^0}$ is the best $s-$sparse estimate of $\mb{\tilde{x}^0}$. Therefore,
\begin{align*}
\norm{\mb{{x}^0} - \mb{x^*}}_2 &=  \norm{\mb{\tilde{x}^0} -\mb{\tilde{x}^0} +\mb{\tilde{x}^0} - \mb{x^*}}_2 \\
& \leq \norm{\mb{{x}^0} -\mb{\tilde{x}^0}}_2 +  \norm{\mb{\tilde{x}^0} - \mb{x^*}}_2 \\
& \leq 2 \norm{\mb{\tilde{x}^0} - \mb{x^*}}_2 \\
& \leq 2 \kappa \norm{\mb{x^*}}_2 = \delta \norm{\mb{x^*}}_2
\end{align*}
with probability at least $1 - 2\left(\frac{Cn}{s}\right)^s\exp\left(-\nu^2s\right)$.
The last inequality follows from the fact that $\mb{\tilde{x}^0}$ is closer to $\mb{x^0}$ compared to $\mb{x^*}$, because $\mb{x^0}$ is the best $s-$sparse approximation of $\mb{\tilde{x}^0}$. $\qed$}

\subsection{Analysis of descent stage}
We perform alternating-minimization as described in~\ref{alg:MoRAM}, starting with $\mb{x^0}$ calculated as above. However, for simplicity, we limit our analysis of the convergence to only one AltMin iteration. In fact, according to our theoretical analysis, if initialized closely enough, one iteration of AltMin suffices for exact signal recovery with sufficiently many measurements. However, in practice this level of close initialization is not often available and we have observed that our algorithm requires more than one AltMin iterations. % to converge to the optimum solution.

The first step is to obtain the initial guess of the bin-index vector (say $\mb{p^0}$) using $\mb{x^0}$.
$$
\mathbf{{p}^{0}} = \frac{\mathbf{1}-\sgn(\langle \mathbf{A} \cdot \mathbf{x^0} \rangle)}{2}.
$$
If we try to undo the effect of modulo operation by adding back $R$ for the affected measurements based on the bin-index vector $\mb{p^0}$, it would introduce an additive error equal to $R$ corresponding to each of the incorrect bin-indices in $\mb{p^0}$.
$$
\mathbf{y^0_c} = \langle \mathbf{A}\mathbf{x^{0}} \rangle = \mathbf{y} - \mathbf{p^0}R,
$$

We show the guaranteed recovery of the true signal as the corruption in the first set of corrected measurements $\mb{y_c^0}$ can be modeled as sparse vector with sparsity less than or equal to $\lambda m$, with $c$ being a fraction that can be explicitly bounded.

To prove this, we first introduce the concept of \emph{binary $\epsilon$-stable embedding} as proposed by~\cite{Jacques2013}. Let $\mathcal{B}^m$ be a Boolean cube defined as $\mathcal{B}^m := \{-1,1\}^m$ and let $S^{n-1}:=\{\mb{x}\in \R^n : \norm{\mb{x}}_2 = 1\}$ be the unit hyper-sphere of dimension $n$.

\definition[Binary $\epsilon$-Stable Embedding]
{A mapping $F: \R^n \rightarrow \mathcal{B}^m$ is a binary $\epsilon$-stable embedding (B$\epsilon$SE) of order $s$ for sparse vectors if:
	$$
	d_S(\mb{x,y}) - \epsilon \leq d_H(F\mb{(x)},F\mb{(y)}) \leq d_S(\mb{x,y}) + \epsilon;
	$$
	for all $\mb{x,y} \in S^{n-1}$ with $|supp(x) \cup supp(y)|\leq s$.
}

In our case, let us define the mapping $F: \R^n \rightarrow \mathcal{B}^m$ as:
$$
F(\mb{x}):= \sgn(\mb{Ax});
$$
with $\mb{A} \sim \mathcal{N}^{m \times n}(0,1)$. We obtain:

\lemma{Let $\mb{A}$ be the matrix generated as $\mb{A} \sim \mathcal{N}^{m \times n}(0,1)$ and suppose $\mb{x^*, x^0} \in \R^n$ are $s-$sparse vectors satisfying $\norm{\mb{x^* - x^0}}_2 \leq \delta\norm{\mb{x^*}}_2$. Let $\eta \in [0,1],~\eps > 0$. If the number of measurements \\
	$m \geq \frac{2}{\eps^2}\left(s\log{(n)} + 2s\log{\left(\frac{35}{\eps}\right)}+\log{\left(\frac{2}{\eta}\right)}\right)$, then, the following is true with probability exceeding $1 - \eta$:
	
	$$
	d_H(\sgn({\mb{Ax^*}}), \sgn({\mb{Ax^0}})) \leq \frac{\delta}{2} + \eps;
	$$
where $d_H$ is Hamming distance between binary vectors defined as:
$$
d_H(\mb{a,b}) := \frac{1}{n}\sum_{i=1}^{n}a_i \oplus b_i,
$$
for $n-$ dimensional binary vectors $\mb{a,b}$.
\label{lemma1} 
} 

\proof{ 
Given $m \geq \frac{2}{\eps^2}\left(s\log{(n)} + 2s\log{\left(\frac{35}{\eps}\right)}+\log{\left(\frac{2}{\eta}\right)}\right)$, using Theorem 3 from~\cite{Jacques2013} we conclude that $F(\cdot)$ is a B$\eps$SE for $s$-sparse vectors.
Thus for sparse vectors $\mb{x^*, x^0}$:
\begin{align}
\label{eq:th3}
d_H(F\mb{(x^*)},F\mb{(x^0)}) \leq d_S(\mb{x^*,x^0}) + \eps.
\end{align}

Here, $d_S(\cdot)$ is defined as the natural angle formed by two vectors. Specifically, for $\mb{p,q}$ in unit norm ball,
$$
d_s(\mb{p},\mb{q}) := \frac{1}{\pi}\textrm{arccos}\langle\mb{p},\mb{q}\rangle = \frac{1}{\pi}\theta,
$$
where $\theta$ is the angle between two unit norm vectors $\mb{p}$ and $\mb{q}$.

We note that,
$$
\norm{\mb{p-q}}_2 = 2\sin(\frac{\theta}{2}).
$$
Thus,
\begin{align}
\label{eq:dhds}
2d_s(\mb{p},\mb{q}) \leq \norm{\mb{p - q}}_2 \leq \pi d_s(\mb{p},\mb{q})
\end{align}
Combining eq.~\ref{eq:th3} and eq.~\ref{eq:dhds}, we conclude,

\begin{align}
d_H(\sgn({\mb{Ax^*}}), \sgn({\mb{Ax^0}})) &\leq \frac{1}{2}\norm{\mb{x^*-x^0}} + \eps,~~\text{i.e.,}  \nonumber \\
d_H(\sgn({\mb{Ax^*}}), \sgn({\mb{Ax^0}})) &\leq \frac{\delta}{2} + \eps. \label{eq:hd}
\end{align}

}

We use this to obtain:

\theorem{Given an initialization $\mb{x^0}$ satisfying $\norm{\mb{x^* - x^0}}_2 \leq \delta\norm{\mb{x^*}}_2$, for $0 < \delta < 1, \eta \in [0,1],~\eps > 0$, if we have number of (Gaussian) measurements satisfying $m \geq \frac{2}{\eps^2}\left(s\log{(n)} + 2s\log{\left(\frac{35}{\eps}\right)}+\log{\left(\frac{2}{\eta}\right)}\right)$ and $s \leq \gamma m/\left(\log\left(n/m\right)+1\right)$, then the estimate after the first iteration $\mb{x^1}$ of Algorithm \ref{alg:MoRAM} is exactly equal to the true signal $\mathbf{x^*}$ with probability at least $1-K\exp(-cm)-\eta$, with $K$ and $c$ being numerical constants.
}

\proof{In the estimation step, Algorithm \ref{alg:MoRAM} dubs the problem of recovering the true signal $\mb{x^*}$ from the modulo measurements as the special case of signal recovery from sparsely corrupted compressive measurements. As we discussed in Section \ref{sec:modeff}, the presence of modulo operation modifies the compressive measurements by adding a constant noise of the value $R$ in fraction of total measurements. However, once we identify correct bin-index for some of the measurements using $\mb{x^0}$, the remaining noise can be modeled as sparse corruptions $\mb{d}$ according to the formulation:

$$
\mb{y} = \mb{Ax} + \mb{I_nR(p^0-p^*)} = \mb{Ax} + \mb{d}.
$$  
Here, the $\sl{l}0$-norm of $\mb{d}$ gives us the number of noisy measurements in $\mb{y^0_c}$.
%$$
%\textnormal{the number of noisy measurements in~} \mb{y^0_c} = \norm{\mb{d}}_0
%$$

If the initial bin-index vector $\mb{p^0}$ is close to the true bin-index vector $\mb{p^*}$, then $\norm{\mb{d}}_0$ is small enough with respect to total number of measurements $m$; thus, $\mb{\mb{d}}$ can be treated as sparse corruption. If we model this corruption as a sparse noise, then we can employ JP for a guaranteed recovery of the true signal given (i) sparsity of the noise is a fraction of total number of measurements; (ii) sufficiently large number of measurements are available.  

We compute $\norm{\mb{d}}_0$ as,

$$
\norm{\mb{d}}_0 =  \norm{(\mb{p^*-p^0})R}_0;
$$

expanding further,
\begin{align*}
\norm{\mb{d}}_0  & = \frac{\mathbf{1}-\sgn(\langle \mathbf{A} \cdot \mathbf{x^0} \rangle)}{2} -  \frac{\mathbf{1}-\sgn(\langle \mathbf{A} \cdot \mathbf{x^*} \rangle)}{2} \\
& = \frac{\sgn({\mb{Ax^*}})- \sgn({\mb{Ax^0}})}{2} \\
& = \frac{F\mb{(x^*)} - F\mb{(x^0)}}{2} \\
& = d_H(F\mb{(x^*)}, F\mb{(x^0)}). \\
\textnormal{From eq.~\ref{eq:hd},} \\
& \leq \frac{\delta}{2} + \epsilon = \gamma m.
\end{align*}

Algorithm \ref{alg:MoRAM} is essentially the Justice Pursuit (JP) formulation as described in \cite{Laska2009}. Exact signal recovery from sparsely corrupted measurements is a well-studied domain with uniform recovery guarantees available in the existing literature. We use the guarantee proved in \cite{li2013compressed} for Gaussian random measurement matrix, which states that one can recover a sparse signal exactly by tractable $\ell_1$-minimization even if a positive fraction of the measurements are arbitrarily corrupted. With $\norm{\mb{d}}_0 \leq \gamma m$, we invoke Theorem $1.1$ from \cite{li2013compressed} to complete the proof.
$\qed$}

\subsection{Minimum number of measurements required}
In this section, we combine the various inequalities involving the number of measurements $m$ from the above derivations to determine the minimum value of $m$ required for successful recovery of the true signal.

First, from theorem \ref{th:1}, we have:
\begin{align}
N \geq C\left(\frac{\nu}{\kappa}\right)^2s.
\label{eq:ineq1}
\end{align}

Next, from Lemma \ref{lemma1}, we get:
\begin{align}
m \geq \frac{2}{\eps^2}\left(s\log{(n)} + 2s\log{\left(\frac{35}{\eps}\right)}+\log{\left(\frac{2}{\eta}\right)}\right).
\label{eq:ineq2}
\end{align}

Now for $R> \rho$, from the analysis in Section~\ref{meascorr}, we have:
\begin{align}
N \geq m \left(1-2\frac{\phi(R-\rho)}{(R-\rho)} \right).
\label{eq:ineq3}
\end{align}

Combining eqs.~\ref{eq:ineq1},\ref{eq:ineq2} \& \ref{eq:ineq3}, we get the minimum number of measurements required as,

\begin{align}
m \geq \max & \Bigg[ C\left(\frac{\nu}{\kappa}\right)^2s \left(1-2\frac{\phi(R-\rho)}{(R-\rho)} \right)^{-1}, \nonumber \\ 
& \frac{2}{\eps^2}\left(s\log{(n)} + 2s\log{\left(\frac{35}{\eps}\right)}+\log{\left(\frac{2}{\eta}\right)}\right)\Bigg].
\label{eq:ineq4}
\end{align}
For $R=4$ and $\rho =3$, the above inequality approximates to:
\begin{align*}
m \geq \max & \Bigg[ 2C\left(\frac{\nu}{\kappa}\right)^2s, \nonumber \\ 
& \frac{2}{\eps^2}\left(s\log{(n)} + 2s\log{\left(\frac{35}{\eps}\right)}+\log{\left(\frac{2}{\eta}\right)}\right)\Bigg].
\end{align*}
%Similarly for the case when $R<\rho$ we get,
%\begin{align}
%m \geq \max &  \Bigg[C\left(\frac{t}{\kappa}\right)^2s \left(\frac{2R}{(1+R^2)} \phi(R) - \frac{\phi(\rho)}{\rho}\right)^{-1}, \nonumber \\ 
%& \frac{2}{\eps^2}\left(s\log{(n)} + 2s\log{\left(\frac{35}{\eps}\right)}+\log{\left(\frac{2}{\eta}\right)}\right) \Bigg]
%\label{eq:ineq5}
%\end{align}
%With $R=1$ and $\rho =3$, we can approximate it as,
%\begin{align*}
%m \geq \max &  \Bigg[4C\left(\frac{t}{\kappa}\right)^2s, \nonumber \\ 
%& \frac{2}{\eps^2}\left(s\log{(n)} + 2s\log{\left(\frac{35}{\eps}\right)}+\log{\left(\frac{2}{\eta}\right)}\right) \Bigg]
%\end{align*}