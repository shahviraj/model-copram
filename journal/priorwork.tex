\section{Prior work}
\label{sec:prior}

We now provide a brief overview of related prior work. At a high level, our algorithmic development follows two (hitherto disconnected) streams of work in the signal processing literature.

\subsubsection*{Phase retrieval} As stated earlier, in this paper we borrow algorithmic ideas from previously proposed solutions for phase retrieval to solve the modulo recovery problem. Being a classical problem with a variety of applications, phase retrieval has been studied significantly in past few years. Approaches to solve this problem can be broadly classified into two categories: convex and non-convex. 

Convex approaches usually consist of solving a constrained optimization problem after lifting the true signal $\mb{x^*}$ in higher dimensional space. The seminal PhaseLift formulation \cite{candes2013phaselift} and its variations \cite{gross2017improved}, \cite{candes2015phasediff} come under this category. Typical non-convex approaches involve finding a good initialization, followed by iterative minimization of a loss function. Approaches based on Wirtinger Flow \cite{candes2015phase, zhang2016reshaped,  chen2015solving, cai2016optimal} and Amplitude flow \cite{wang2016sparse,wang2016solving} come under this category. 

In recent works, extending phase retrieval algorithms to situations where the underlying signal exhibits a sparse representation in some known basis has attracted interest. Convex approaches for sparse phase retrieval include \cite{ohlsson2012cprl, li2013sparse,bahmani2015efficient,jaganathan2012recovery}. Similarly, non-convex approaches for sparse phase retrieval include \cite{netrapalli2013phase, cai2016optimal, wang2016sparse}. Our approach in this paper towards solving the modulo recovery problem can be viewed as a complement to the non-convex sparse phase retrieval framework advocated in \cite{Jagatap2017}. 

\subsubsection*{Modulo recovery} The modulo recovery problem is also known in the classical signal processing literature
as phase unwrapping. The algorithm proposed in \cite{bioucas2007phase} is specialized to images, and employs graph cuts for phase unwrapping from a single modulo measurement per pixel. However, the inherent assumption there is that the input image has very few sharp discontinuities, and this makes it unsuitable for practical situations with textured images. Our work is motivated by the recent work of \cite{ICCP15_Zhao} on high dynamic range (HDR) imaging using a modulo camera sensor. For image reconstruction using multiple measurements, they propose the multi-shot UHDR recovery algorithm, with follow-ups developed further in \cite{Lang2017}. However, the multi-shot approach depends on carefully designed camera exposures, while our approach succeeds for non-designed (generic) linear observations; moreover, they do not include sparsity in their model reconstructions. In our previous work \cite{Shah}, we proposed a different extension based on \cite{ICCP15_Zhao, soltani2017stable} for signal recovery from quantized modulo measurements, which can also be adapted for sparse measurements, but there too the measurements need to be carefully designed.

In recent literature, several authors have attempted to theoretically understand the modulo recovery problem. Given modulo-transformed time-domain samples of a band-limited function, \cite{Bhandari} provides a stable algorithm for perfect recovery of the signal and also proves sufficiency conditions that guarantees the perfect recovery. \cite{Cucuringu2017} formulates and solves an QCQP problem with non-convex constraints for denoising the modulo-1 samples of the unknown function along with providing a least-square based modulo recovery algorithm.. However, both these methods relay on the smoothness of the band-limited function as a prior structure on the signal, and as such it is unclear how to extend their use to more complex modeling priors (such as sparsity in a given basis). 

For a qualitative comparison of our MoRAM method with existing approaches, refer Table~\ref{tab:comp}. The table suggests that the previous approaches varied from the Nyquist-Shannon sampling setup only along the amplitude dimension, as they rely on band-limitedness of the signal and uniform sampling grid. We vary the sampling setup along both the amplitude and time dimensions by incorporating sparsity in our model, which enables us to work with non-uniform sampling grid (random measurements) and achieve a provable sub-Nyquist sample complexity.

\begin{table*}[t]
	\centering
	\caption{Comparison of MoRAM with existing modulo recovery methods. \label{tab:comp}}

	\begin{tabular}{|p{4.7cm}|p{3cm}|p{2.2cm}|p{2.7cm}|p{3.22cm}|}
		\cline{1-5}
		& Unlimited Sampling~\cite{Bhandari}& OLS Method~\cite{Cucuringu2017} & multishot UHDR~\cite{ICCP15_Zhao}  & MoRAM (our approach) \\ \cline{1-5}
		
		Assumption on structure of signal  & Bandlimited   & Bandlimited &  No assumptions & Sparsity     \\ \cline{1-5}
		Sampling scheme  & uniform grid &  uniform grid  & (carefully chosen) linear measurements   & 
		random linear measurements \\ \cline{1-5}
		Sample complexity & oversampled, $\mathcal{O}(n)$    & --    & oversampled, $\mathcal{O}(n)$   & undersampled,~$\mathcal{O}(s\log(n))$ \\ \cline{1-5}
		Provides sample complexity bounds?  & Yes  & --  & No & Yes \\ \cline{1-5}
		Leverages Sparsity?  & No & No   & No    & Yes \\ \cline{1-5}
		(Theoretical) bound on dynamic range    & Unbounded   & Unbounded     & Unbounded   & $2R$  \\ \cline{1-5}                 
	\end{tabular}	
	\label{tab:compare}
\end{table*}
