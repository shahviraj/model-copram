\section{Prior work}
\label{sec:prior}

\subsubsection*{Phase retrieval} As stated earlier, in this paper we borrow ideas from the solutions of phase retrieval problem to solve modulo recovery problem. Being a classical problem with variety of applications, phase retrieval has been studied significantly in past few years. Approaches to solve this problem can be broadly classified into two categories: convex and non-convex. 
Convex approach usually consist of solving a constraint optimization problem after lifting the true signal $\mb{x^*}$ in higher dimensional space to linearize the problem. PhaseLift algorithm \cite{candes2013phaselift} and its variations \cite{gross2017improved}, \cite{candes2015phasediff} come under this category. Typical non-convex approaches involve finding a good initialization followed by iterative minimization of a loss function. Approaches based on Amplitude flow \cite{wang2016sparse,wang2016solving} and Wirtinger flow \cite{candes2015phase, zhang2016reshaped,  chen2015solving, cai2016optimal} come under this category. 
In recent works, phase retrieval problem for the cases where underlying signal is sparse is of growing interest. Some of the convex approaches for sparse phase retrieval includes \cite{ohlsson2012cprl, li2013sparse,bahmani2015efficient,jaganathan2012recovery}. Similarly, non-convex approaches for sparse phase retrieval includes \cite{netrapalli2013phase, cai2016optimal, wang2016sparse}. Our approach in this paper towards solving the modulo recovery problem is mainly inspired from the non-convex sparse phase retrieval framework advocated in \cite{Jagatap2017}. 

\subsubsection*{Modulo recovery} The modulo recovery problem is also known in the literature
as phase unwrapping. The algorithm proposed in \cite{bioucas2007phase} is specialized to images, and employs graph cuts for phase unwrapping from a single modulo measurement per pixel. However, the inherent assumption there is that the input image has very few sharp discontinuities, and this makes it unsuitable for practical situations with textured images. Our main motivation for this paper is the work of \cite{ICCP15_Zhao} on HDR imaging using a modulo camera sensor. For image reconstruction using multiple measurements, it proposes the method called the multi-shot UHDR recovery algorithm. The more robust version of multi-shot UHDR is further proposed in \cite{Lang2017}. However, both these methods depend on carefully decided camera exposures and don't include sparsity in their models. In our previous work \cite{Shah}, we proposed an algorithm based on \cite{ICCP15_Zhao, soltani2017stable} for signal recovery from quantized modulo measurements, which can also be adapted for sparse measurements. 

In recent literature, there have been several attempts to come with strong theoretical foundation pertaining to modulo sampling and recovery.  Given the modulo samples of a bandlimited function, \cite{Bhandari} provides a stable algorithm for perfect recovery of the signal and also proves sufficiency conditions that guarantees the perfect recovery. \cite{Cucuringu2017} formulates and solves an QCQP problem with non-convex constraints for recovering the correct samples of the unknown function from its modulo 1 ($R =1$) samples. However, both these methods relay on the smoothness of the bandlimited function as a prior structure on the signal, and are restricted to be used for recovering samples of a smooth bandlimited function, not the signal or image. \todo{needs more clarity on the difference between Bhandari et al and our setup}.
