\section{Mathematical model}
\label{sec:model}
Let us introduce some notation. We denote matrices using bold capital-case letters ($\mb{A,B}$), column vectors using bold-small case letters ($\mb{x,y,z}$ etc.) and scalars using non-bold letters ($R,m$ etc.). The superscript $\t$ to denotes the transpose. The cardinality of set $S$ is given by $\card(S)$. The signum function is defined as $\sgn(x) := \frac{x}{|x|}, \forall x \in \R, x \neq 0$, with $\sgn(0)=1$. An $i^{th}$ element of the vector $\mb{x} \in \R^n$ is denoted by $x_{i}$. Similarly, $i^{th}$ row of the matrix $\mb{A} \in \R^{m \times n}$ is denoted by $\mb{a_i}$, while the element of $\mb{A}$ in $i^{th}$ row and $j^{th}$ column is denoted as $a_{ij}$.

As depicted in Fig.~\ref{fig:compare}(a), we consider the modulo operation within 2 periods.
% and assume that the value of dynamic range parameter $R$ is large enough so that all the elements of $\mb{Ax^*}$ are covered within the domain of operation of modulo function.
If we write the modulo function of Fig.~\ref{fig:compare}(a) in terms of a signum function, then the measurement model of Eq.~\ref{eq:modmeas0} becomes, 
\begin{equation}
y_i= \langle \mathbf{a_i} \cdot \mathbf{x^*} \rangle+\left( \frac{1-\sgn(\langle \mathbf{a_i} \cdot \mathbf{x^*} \rangle)}{2}\right)R,~~i = \{1,..,m\}.
\label{eq:modmeas2}
\end{equation} 
If we divide the number line in two bins, then the coefficient of $R$ in above equation can be seen as a bin-index, a binary variable which takes value $0$ when $\sgn(t)=1$, or $1$ when $\sgn(t)=-1$. We denote such bin-index vector as $\mb{p} \in \R^m$. Each element of the true bin-index vector $\mb{p}^*$ is given as,
$$
p^*_i = \frac{1-\sgn(\langle \mathbf{a_i} \cdot \mathbf{x^*} \rangle)}{2},~~i = \{1,..,m\}.
$$

If we ignore the presence of modulo operation in above formulation, then it reduces to a standard compressive sensing problem. In that case, the compressed measurements $y_{c_i}$ would just be equal to $\langle \mathbf{a_i} \cdot \mathbf{x^*} \rangle$. While we have access only to the compressed modulo measurements $\mb{y}$, it is useful to write $\mb{y}$ in terms of true compressed measurements $\mb{y}_c$. Thus,
$$
y_i = \langle \mathbf{a_i} \cdot \mathbf{x^*} \rangle + p^*_iR = y_{c_i}+p^*_iR.
$$

It is evident that if we can recover $\mathbf{p^*}$ successfully, we can calculate the true compressed measurements $\langle \mathbf{a_i} \cdot \mathbf{x^*} \rangle$ and use them to reconstruct $\mathbf{x^*}$ with any sparse recovery algorithm such as CoSaMP or basis-pursuit.

\section{Algorithm and Main Results}
Given $\mathbf{y, A}, s, R$, our approach recovers $\mathbf{x^*}$ and $\mathbf{p^*}$ in two steps: (i) an initialization step, and (ii) descent step via alternating minimization.

\subsection{Initialization by re-calculating the measurements}
\label{sec:init}	
Similar to other non-convex approaches, MoRAM also requires an initial estimate $\mathbf{{x}^0}$ that is close to the true signal $\mathbf{{x}^*}$. We propose a novel initialization method to re-calculate the true Gaussian measurements ($\mb{y_c}= \mb{Ax^*}$) from the available modulo measurements. We undo the effect of modulo operation for the fraction of the total measurements, and calculate the initial estimate using such corrected measurements.
\subsubsection{The effect of modulo operation} 
\label{sec:modeff}
We observe the density plots of the $\mathbf{Ax^*}$(Fig.~\ref{fig:hist1}) and $\mathbf{\mod(\mathbf{Ax^*})}$(Fig.~\ref{fig:hist2}) to understand the information we can obtain from the modulo measurements. We are particularly interested in the case where elements of $\mathbf{Ax^*}$ are low compared to the value of $R$. Here, we approximate the spread of the $\mathbf{Ax^*}$ with a hyper-parameter $\rho$. We choose a value of $\rho$ such that it bounds the maximum element of $|\mathbf{Ax^*}|$ with very high probability. As shown in Fig.~\ref{fig:hist1}, with very high probability, $\mathbf{Ax^*}$ lies within $[-\rho, \rho]$. %In practice, $\rho$ can be calculated based on tail bounds.
In Fig.~\ref{fig:hist2}, we can see the density plots after modulo operation given $R>\rho$. Comparing these distributions with the distribution of true measurements (Fig.~\ref{fig:hist1}), we can observe how the values of first density plot translates into second when the modulo operation is applied. We can draw following conclusions:
\begin{itemize}
	\item Only the values lying on the negative side of the x-axis are going to be affected.
	\item Values lying very close to the origin on the negative side of the x-axis in the first density plot, would now shift by $R$, and would occupy a place very close to $R$ in second plot. For $R>\rho$, this region is shaded with green color in Fig.~\ref{fig:hist2}. Correct bin-index for the elements in $\mathbf{y}$ with value lying between $\rho$ and $R$ is $p^{init}_{i} = 1$.
	\item For $R>\rho$, density plot of the positive region very close to the origin would remain unaffected. This region is shaded with orange color in Fig.~\ref{fig:hist2}. Correct bin-index for all the elements in $\mathbf{y}$ with value lying between $0$ and $R-\rho$ is $p^{init}_{i} = 0$.
	\item Nothing can be concluded for measurements lying in between the above mentioned ranges. This region is shaded with gray color in Fig.~\ref{fig:hist2}. Correct bin-index cannot be identified for this region, so we assign all the elements in $\mathbf{y}$ with value lying between $0$ and $(R-\rho)$ as $p^{init}_{i} = 0$. The lower and upper bounds ($t_l$~\&$~t_u$) of this region of uncertainty can be obtained as, 
	\begin{align*}
	t_l & = R-\rho, \\
	t_u & = R.
	\end{align*}
	%\item Irrespective of relationship between $\rho$ and $R$, all the values lying in the negative half of the real line have the bin-index equal to $1$, and all the values greater than $R$ have the bin-index equal to $0$.
\end{itemize}

\begin{algorithm}[t]
	\caption{\textsc{MoRAM-initialization}}
	\label{alg:RCM}
	\begin{algorithmic}
		\State\textbf{Inputs:} $\mathbf{y}$, $\mathbf{A}$, $s$, $R$, $\rho$
		\State\textbf{Output:}  $\mb{x^0}$
		\State $T \leftarrow \emptyset$, $t_l \leftarrow (R-\rho)$, $t_u \leftarrow R$
		\For {$i= 0:m$}
%		\If {$y_l<0$}
%		\State {$p^{rcm}_l = 1$, $T \leftarrow T \cup {l}$}
%		\ElsIf {$0\leq y_l < t_l$}
%		\State {$p^{rcm}_l = 0$, $T \leftarrow T \cup {l}$}
%		\ElsIf {$t_l\leq y_l < t_u$}
%		\State {$p^{rcm}_l = 0$}
%		\ElsIf {$t_u \leq y_l < R$}
%		\State {$p^{rcm}_l = 1$, $T \leftarrow T \cup {l}$}
%		\ElsIf {$R  \leq y_l$}
%		\State {$p^{rcm}_l = 0$, $T \leftarrow T \cup {l}$}
%		
%		\EndIf
		
		\If {$t_l > y_i$ or $ y_i \geq t_u$}
		\State {$T \leftarrow T \cup \{i\}$}
		\EndIf
		\State Calculate $p^{init}_i$ according to Eq. ~\ref{eq:rcm}.
		\EndFor
		\State $N \leftarrow |T|$
		\State $\mb{x^0} \leftarrow H_s\left( \frac{1}{N}\sum_{i=1}^{N}y_{T,i}a_{T,i}\right)$
	\end{algorithmic}
\end{algorithm}

\begin{figure}[h]
	\begin{center}
		\begin{tikzpicture}[scale=0.7, every node/.style={scale=0.7}]
		\def\normaltwo{\x,{3*1/exp(((\x)^2)/2)}}
		\def\y{4.4}
		
		%\fill [fill=orange!60] (2.6,0) -- plot[domain=0:4.4] (\normaltwo) -- ({\y},0) -- cycle;
		
		% Draw and label normal distribution function
		\draw[color=blue,domain=-4.25:4.25,thick] plot (\normaltwo) node[right] {};
		\draw[<->] (-5,0) -- (5,0) node[right] {$\mathbf{Ax^*}$};
		\draw[<->] (0,-1) -- (0,4);
		
		\draw (-3,-0.5) node(below) {$-\rho$};
		\draw (-4,-0.5) node(below) {$-R$};
		\draw (3,-0.5) node(below) {$\rho$};
		\draw (4,-0.5) node(below) {$R$};
		
		\foreach \x in {-3,-4,3,4}
		{        
			\coordinate (A\x) at ($(0,0)+(\x*1cm,0)$) {};
			\draw ($(A\x)+(0,5pt)$) -- ($(A\x)-(0,5pt)$);
			
		}
		%		\draw (-1.5,-0.5) node(below) {$p_i = 1$};
		%		\draw (0,-1.5) node(right) {$f(t) = \mod(t,R)$};
		%		\draw[scale=0.5,domain=-7:0,smooth,variable=\x,blue, ultra thick] plot ({\x},{\x+4});
		%		\draw[scale=0.5,domain=0:7,smooth,variable=\x,blue, ultra thick]  plot ({\x},{\x});
		\end{tikzpicture}
	\end{center}
	\caption{\emph{Histogram of $\mathbf{Ax^*}$}}
	\label{fig:hist1}
\end{figure}
\begin{figure}[h]
	\begin{center}
		\begin{tikzpicture}[scale=0.7, every node/.style={scale=0.7}]
		\def\normaltwo{\x,{3*1/exp(((\x)^2)/2)}}
		\def\normalone{\x,{3*1/exp(((\x-4)^2)/2)}}
		\def\normalsum{\x,{3*1/exp(((\x-4)^2)/2)+3*1/exp(((\x)^2)/2)}}
		\def\y{3}
		\def\fy{3*1/exp(((\y-4)^2)/2)}
		\fill [fill=orange!60] (0,0) -- plot[domain=0:1] (\normaltwo) -- (1,0) -- cycle;
		\fill [fill=green!60] (3,0) -- plot[domain=3:4] (\normalone) -- (4,0) -- cycle;
		\fill [fill=gray!30] (1,0) -- plot[domain=1:3] (\normalsum) -- (3,0) -- cycle;
		% Draw and label normal distribution function
		\draw[color=blue,domain=-0:3,dashed,thick] plot (\normaltwo) node[right] {};
		\draw[color=blue,domain=1:4,dashed,thick] plot (\normalone) node[right] {};
		\draw[color=blue,domain=-0:4,thick] plot (\normalsum) node[right] {};
		\draw[<->] (-5,0) -- (5,0) node[right] {$\mathbf{Ax^*}$};
		\draw[<->] (0,-1) -- (0,4);
		\draw[dashed] ({\y},{\fy}) -- ({\y},0);
		\draw[dashed] ({4},{3}) -- ({4},0);
		\draw[dashed] ({1},{\fy}) -- ({1},0);
		\draw (-3,-0.5) node(below) {$-\rho$};
		\draw (-4,-0.5) node(below) {$-R$};
		\draw (3,-0.5) node(below) {$\rho$};
		\draw (4,-0.5) node(below) {$R$};
		\draw (1,-0.5) node(below) {$R-\rho$};
		\draw (0.5,1) node(below) {$p_i=0$};
		\draw (3.5,1) node(below) {$p_i=1$};
		\draw (2,0.5) node(below) {$p_i=0$};
		\foreach \x in {-3,-4,1,3,4}
		{        
			\coordinate (A\x) at ($(0,0)+(\x*1cm,0)$) {};
			\draw ($(A\x)+(0,5pt)$) -- ($(A\x)-(0,5pt)$);
			
		}
		%		\draw (-1.5,-0.5) node(below) {$p_i = 1$};
		%		\draw (0,-1.5) node(right) {$f(t) = \mod(t,R)$};
		%		\draw[scale=0.5,domain=-7:0,smooth,variable=\x,blue, ultra thick] plot ({\x},{\x+4});
		%		\draw[scale=0.5,domain=0:7,smooth,variable=\x,blue, ultra thick]  plot ({\x},{\x});
		\end{tikzpicture}
	\end{center}
	\caption{\emph{Histogram of $\mod(\mathbf{Ax^*})$, $R>\rho$}}
	\label{fig:hist2}
\end{figure}
We divide the number line in the following $3$ intervals, and assign the bin-index accordingly:
\begin{equation}
{p}^{init}_{i} = 
\begin{cases}
0,& \text{if } 0\leq y_i < t_l \\
0,& \text{if } t_l\leq y_i < t_u ~~~~ \textnormal{(region of uncertainty)} \\
1,& \text{if } t_u \leq y_i < R \\
\end{cases}
\label{eq:rcm}
\end{equation}
Thus, we can identify the correct bin-index for part of the measurements just by observing their magnitude. We define set $T$ as the set of measurements for which we can identify the bin-index correctly. We introduce $N=: \card(T)$.
\begin{align*}
N & =  m - \card(\textnormal{region of uncertainity})
\end{align*}
Value of $N$ largely depends on the difference between $\rho$ and $R$.
Once we identify the correct bin-index for part of the measurements, we can re-calculate corrected measurements as,
$$
\mathbf{y_{c} = y + p^{init}R}.
$$
We use these corrected measurements $\mathbf{y_{init}}$ to calculate the initial estimate $\mathbf{{x}^0}$ using the estimation step described in the upcoming section.
\subsubsection{Calculating the initial estimate from the corrected measurements}
In the estimation step, $\mb{x^0}$ is calculated only from the $N$ corrected measurements using first order unbiased estimator. For that, we use the versions of $\mb{y}$ and $A$ truncated to the indices belong to set $T$:
\begin{equation}
\mb{x^0} = H_s\left( \frac{1}{N}\sum_{i=1}^{N}y_{T,i}a_{T,i}\right)
\label{eq:init}
\end{equation}
where $H_s$ denotes the hard thresholding operator that keeps the $s$ largest absolute entries of a vector and sets the other entries to zero.
\subsection{Alternating Minimization}
\label{sec:altmin}
\begin{algorithm}[H]
	\caption{\textsc{MoRAM-descent}}
	\label{alg:MoRAM}
	\begin{algorithmic}
		\State\textbf{Inputs:} $\mathbf{y}$, $\mathbf{A}$, $s$, $R$
		\State\textbf{Output:}  $\widehat{x}$
		\State $m,n \leftarrow \mathrm{size}(\mathbf{A})$ 
		\State \textbf{Initialization}
		\State $\mathbf{x^0} \leftarrow \textrm{MoRAM-initialization}(\mathbf{y, A})$ 
		\State \textbf{Alternating Minimization}
		\For {$l =0:L$}
		\State $\mathbf{{p}^{t}} \leftarrow \frac{\mathbf{1}-\sgn(\langle \mathbf{A} \cdot \mathbf{x^t} \rangle)}{2}$
		\State $\mathbf{y^t_c} \leftarrow \mathbf{y} - \mathbf{p^t}R$
		\State $\mathbf{{x}^{t+1}}\leftarrow \argmin_{\mb{u}=[\mathbf{x~d}]^\t}\norm{u}_1  s.t.~ \begin{bmatrix} \mathbf{A} & \mathbf{I} \end{bmatrix}\mb{u} = \mathbf{y}$ 
		\State $\implies \mathbf{{x}^{t+1}}\leftarrow \small{JP(\frac{1}{\sqrt{m}}\begin{bmatrix} \mathbf{A} & \mathbf{I} \end{bmatrix},\frac{1}{\sqrt{m}}\mathbf{y},[\mathbf{x^t~~p^t}]^\t)}$.
		\EndFor
	\end{algorithmic}
\end{algorithm}

Using Eq.~\ref{eq:init}, we calculate the initial estimate of the signal $\mathbf{{x}^0}$ which is relatively close to the true vector $\mathbf{x^*}$. Starting with $\mathbf{{x}^0}$, we  calculate the estimates of $\mathbf{p}$ and $\mathbf{x}$ in alternating fashion to converge to the original signal $\mathbf{x^*}$. At each iteration of our Alternating Minimization, we use the current estimate of the signal ${\mathbf{x^t}}$ to get the value of the bin-index vector $\mathbf{{p}^t}$ as following:
\begin{equation}
\mathbf{{p}^{t}} = \frac{\mathbf{1}-\sgn(\langle \mathbf{A} \cdot \mathbf{x^t} \rangle)}{2}.
\label{step1}
\end{equation}

Given $\mathbf{x^0}$ is close to $\mathbf{x^*}$, $\mathbf{p^0}$ would also be close to $\mathbf{p^*}$. Ideal way is to calculate the correct compressed measurements $\mathbf{y^t_c}$ using $\mathbf{p^t}$, and use $\mathbf{y^t_c}$ with any popular compressive recovery algorithms such as CoSaMP or basis pursuit to calculate the next estimate $\mathbf{{x}_{t+1}}$. Thus,


$$
\mathbf{y^t_c} = \langle \mathbf{A}\mathbf{x^{t}} \rangle = \mathbf{y} - \mathbf{p^t}R,
$$
$$
\mathbf{{x}^{t+1}} = \argmin_{\mathbf{x} \in \mathcal{M}_s}\norm{\mathbf{Ax} - \mathbf{y^t_c}}_2^2, %~~\mathrm{s.to}~~x^* \in \mathcal{M}_s,
$$

However, it should be noted that even the small error $\mathbf{d^t} = \mathbf{p^t - p^*}$ would reflect heavily in the calculation of $\mathbf{y^t_c}$, as each incorrect bin-index would add a noise of the magnitude $R$ in $\mathbf{y^t_c}$. Experiments suggest that the typical sparse recovery algorithms are not robust enough to cope up with such large errors in $\mathbf{y^t_c}$ \cite{Laska2009}. To tackle this issue, we employ the justice pursuit based formulation which is specifically robust towards grossly corrupted measurements. We consider the fact that the nature of error $\mathbf{d^t}$ is sparse; and each erroneous element of $\mathbf{p}$ adds a noise of the magnitude $R$ in $\mathbf{y^t_c}$.

The augmented optimization problem becomes,

$$
\mathbf{{x}^{t+1}}=\argmin_{[\mathbf{x~d}]^\t \in \mathcal{M}_{s+s_p}}\norm{\begin{bmatrix} \mathbf{A} & \mathbf{I} \end{bmatrix} \begin{bmatrix} \mathbf{x} \\ \mathbf{d} \end{bmatrix} - \mathbf{y^t_c}}_2^2, %~~\mathrm{s.to}~~x^* \in \mathcal{M}_s,
$$
%\begin{equation}
%= \cosamp(\frac{1}{\sqrt{m}}\begin{bmatrix} \mathbf{A} & \mathbf{I} \end{bmatrix},\frac{1}{\sqrt{m}}\mathbf{y_c^t},s+s_p,[\mathbf{x^t~~p^t}]^\t).
%\label{eq:robcosamp}
%\end{equation}

However, the sparsity of $\mathbf{d^t}$ is unknown, suggesting that the sparse recovery algorithms taking sparsity as a parameter cannot be used here. Thus, as we employ basis pursuit, which doesn't rely on sparsity. The robust formulation of basis pursuit is referred as Justice Pursuit (JP) \cite{Laska2009} in the literature, specified in Eq.~\ref{eq:jp}.
\begin{equation}
\implies \mathbf{{x^{t+1}}} = JP(\frac{1}{\sqrt{m}}\begin{bmatrix} \mathbf{A} & \mathbf{I} \end{bmatrix},\frac{1}{\sqrt{m}}\mathbf{y^t_c},[\mathbf{x^t~~p^t}]^\t).
\label{eq:jp}
\end{equation}
We repeat the steps of bin-index calculation (as in Eq.~\ref{step1}) and sparse recovery (Eq.~\ref{eq:jp}) alternatively for $\mathrm{N}$ iterations. Our algorithm is able to achieve perfect convergence, as supported by the results in the experiments section.
