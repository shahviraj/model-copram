\section{Mathematical model}
\label{sec:model}
Let us introduce some notation. We denote matrices using bold capital-case letters ($\mb{A,B}$), column vectors using bold-small case letters ($\mb{x,y,z}$ etc.) and scalars using non-bold letters ($R,m$ etc.). %The superscript $\t$ to denotes the transpose. 
The cardinality of set $S$ is given by $\card(S)$. The signum function is defined as $\sgn(x) := \frac{x}{|x|}, \forall x \in \R, x \neq 0$, with $\sgn(0)=1$. %An $i^{th}$ element of the vector $\mb{x} \in \R^n$ is denoted by $x_{i}$.% Similarly, $i^{th}$ row of the matrix $\mb{A} \in \R^{m \times n}$ is denoted by $\mb{a_i}$, while the element of $\mb{A}$ in $i^{th}$ row and $j^{th}$ column is denoted as $a_{ij}$. 
The projection of vector $\mb{x} \in \R^n$ onto a set of coordinates $S$ is represented as $\mb{x}_S \in \R^n,~\mb{x}_{S_j} = \mb{x}_j$ for $j \in S$, and $0$ elsewhere.

As depicted in Fig.~\ref{fig:compare}(a), we consider the modulo operation within 2 periods.
% and assume that the value of dynamic range parameter $R$ is large enough so that all the elements of $\mb{Ax^*}$ are covered within the domain of operation of modulo function.
If we write the modulo function of Fig.~\ref{fig:compare}(a) in terms of a signum function, then the measurement model of Eq.~\ref{eq:modmeas0} becomes, 
\begin{equation}
y_i= \langle \mathbf{a_i} \cdot \mathbf{x^*} \rangle+\left( \frac{1-\sgn(\langle \mathbf{a_i} \cdot \mathbf{x^*} \rangle)}{2}\right)R,~~i = \{1,..,m\}.
\label{eq:modmeas2}
\end{equation} 
If we divide the number line in two bins, then the coefficient of $R$ in above equation can be seen as a bin-index, a binary variable which takes value $0$ when $\sgn(t)=1$, or $1$ when $\sgn(t)=-1$. We denote such bin-index vector as $\mb{p} \in \R^m$. Each element of the true bin-index vector $\mb{p}^*$ is given as,
$$
p^*_i = \frac{1-\sgn(\langle \mathbf{a_i} \cdot \mathbf{x^*} \rangle)}{2},~~i = \{1,..,m\}.
$$

If we ignore the presence of modulo operation in above formulation, then it reduces to a standard compressive sensing problem. In that case, the compressed measurements $y_{c_i}$ would just be equal to $\langle \mathbf{a_i} \cdot \mathbf{x^*} \rangle$. While we have access only to the compressed modulo measurements $\mb{y}$, it is useful to write $\mb{y}$ in terms of true compressed measurements $\mb{y}_c$. Thus,
$$
y_i = \langle \mathbf{a_i} \cdot \mathbf{x^*} \rangle + p^*_iR = y_{c_i}+p^*_iR.
$$

It is evident that if we can recover $\mathbf{p^*}$ successfully, we can calculate the true compressed measurements $\langle \mathbf{a_i} \cdot \mathbf{x^*} \rangle$ and use them to reconstruct $\mathbf{x^*}$ with any sparse recovery algorithm such as CoSaMP~\cite{needell2010cosamp} or basis-pursuit~\cite{chen2001atomic,spgl1:2007,BergFriedlander:2008}.

\section{Algorithm and Main Results}
Given $\mathbf{y, A}, s, R$, our approach recovers $\mathbf{x^*}$ and $\mathbf{p^*}$ in two steps: (i) an initialization step, and (ii) descent step via Alt-Min.

\subsection{Initialization by re-calculating the measurements}
\label{sec:init}	
Similar to other non-convex approaches, MoRAM also requires an initial estimate $\mathbf{{x}^0}$ that is close to the true signal $\mathbf{{x}^*}$.
% We propose a novel initialization method to re-calculate the true Gaussian measurements ($\mb{y_c}= \mb{Ax^*}$) from the available modulo measurements. We undo the effect of modulo operation for the fraction of the total measurements, and calculate the initial estimate using such corrected measurements.

To do so, we first introduce a hyper-parameter $\rho$ that approximates the spread of the entries of $\mathbf{Ax^*}$. We choose a value of $\rho$ such that it bounds the maximum element of $|\mathbf{Ax^*}|$ with very high probability. As shown in Fig.~\ref{fig:densityplot}, with very high probability, $\mathbf{Ax^*}$ lies within $[-\rho, \rho]$. In Fig.~\ref{fig:densityplot} (a) and (b), we provide the density plots of the $\mb{y_c} = \mathbf{Ax^*}$ and $\mb{y} = \mathbf{\mod(\mathbf{Ax^*})}$ respectively. Note that the compressed measurements $\mathbf{y_c}$ follow the standard normal distribution, as $\mb{A}$ is Gaussian random matrix. These plots essentially depict the distribution of our observations before and after the modulo operation.%, and help us to understand the transformation effected by modulo operation. %In practice, $\rho$ can be calculated based on tail bounds.

With reference to Fig.~\ref{fig:densityplot}(a), we divide the compressed observations $\mb{y_c}$ in two sets: $\mb{y_{c,+}}$ contains all the positive observations lying in $[0,\rho]$ with bin-index$=0$ (orange), while $\mb{y_{c,-}}$ contains all the negative ones in $[-\rho,0]$ with bin-index$=1$ (green). Similarly, in Fig.~\ref{fig:densityplot}(b), modulo observations $\mb{y}$ can also be divided in two sets $\mb{y}_+ = \mod(\mb{y}_{c,+})$ and $\mb{y}_- = \mod(\mb{y}_{c,-})$. Modulo function shifts the set $\mb{y}_{c,-}$ to the right by $R$, while leaves the set $\mb{y}_{c,+}$ unaffected. Thus, the sets $\mb{y}_-$ and $\mb{y}_+$ occupy the interval $[R-\rho,R]$ and $[0,\rho]$ respectively. From that, it can be concluded with surety that all the modulo observations lying in the interval $[\rho, R]$ (green) are part of set $\mb{y}_{c,-}$, and have bin-index value equal to $1$. Similarly, all the modulo observations lying in the interval $[0,R-\rho]$ (orange) are part of set $\mb{y}_{c,+}$, and have bin-index value equal to $0$. Nothing can be concluded for other modulo observations lying in $[R-\rho,\rho]$ (gray, region of uncertainty). We assign bin-index$=0$ for such observations. Thus, the corrected bin-index vector can be calculated as:
\begin{equation}
{p}^{init}_{i} = 
\begin{cases}
0,& \text{if } 0\leq y_i < (R-\rho) \\
0,& \text{if } (R-\rho)\leq y_i < \rho ~ \textnormal{(region of uncertainty)} \\
1,& \text{if } \rho \leq y_i < R \\
\end{cases}
\label{eq:rcm}
\end{equation}
%\begin{itemize}
%	\item Only the values lying on the negative side of the x-axis are going to be affected.
%	\item Values lying very close to the origin on the negative side of the x-axis in the first density plot, would now shift by $R$, and would occupy a place very close to $R$ in second plot. For $R>\rho$, this region is shaded with green color in Fig.~\ref{fig:hist2}. Correct bin-index for the elements in $\mathbf{y}$ with value lying between $\rho$ and $R$ is $p^{init}_{i} = 1$.
%	\item For $R>\rho$, density plot of the positive region very close to the origin would remain unaffected. This region is shaded with orange color in Fig.~\ref{fig:hist2}. Correct bin-index for all the elements in $\mathbf{y}$ with value lying between $0$ and $R-\rho$ is $p^{init}_{i} = 0$.
%	\item Nothing can be concluded for measurements lying in between the above mentioned ranges. This region is shaded with gray color in Fig.~\ref{fig:hist2}. Correct bin-index cannot be identified for this region, so we assign all the elements in $\mathbf{y}$ with value lying between $0$ and $(R-\rho)$ as $p^{init}_{i} = 0$. The lower and upper bounds ($t_l$~\&$~t_u$) of this region of uncertainty can be obtained as, 
%	\begin{align*}
%	t_l & = R-\rho, \\
%	t_u & = R.
%	\end{align*}
%	%\item Irrespective of relationship between $\rho$ and $R$, all the values lying in the negative half of the real line have the bin-index equal to $1$, and all the values greater than $R$ have the bin-index equal to $0$.
%\end{itemize}

\begin{algorithm}[t]
	\caption{\textsc{MoRAM-initialization}}
	\label{alg:RCM}
	\begin{algorithmic}
		\State\textbf{Inputs:} $\mathbf{y}$, $\mathbf{A}$, $s$, $R$, $\rho$
		\State\textbf{Output:}  $\mb{x^0}$
		\State $T \leftarrow \emptyset$%, $t_l \leftarrow (R-\rho)$, $t_u \leftarrow R$
		\For {$i= 0:m$}
%		\If {$y_l<0$}
%		\State {$p^{rcm}_l = 1$, $T \leftarrow T \cup {l}$}
%		\ElsIf {$0\leq y_l < t_l$}
%		\State {$p^{rcm}_l = 0$, $T \leftarrow T \cup {l}$}
%		\ElsIf {$t_l\leq y_l < t_u$}
%		\State {$p^{rcm}_l = 0$}
%		\ElsIf {$t_u \leq y_l < R$}
%		\State {$p^{rcm}_l = 1$, $T \leftarrow T \cup {l}$}
%		\ElsIf {$R  \leq y_l$}
%		\State {$p^{rcm}_l = 0$, $T \leftarrow T \cup {l}$}
%		
%		\EndIf
		
		\If {$(R-\rho) > y_i$ or $ y_i \geq \rho$}
		\State {$T \leftarrow T \cup \{i\}$}
		\EndIf
		\State Calculate $p^{init}_i$ according to Eq.~\ref{eq:rcm}.
		\EndFor
		\State $N \leftarrow |T|$, calculate $\mb{y^{init}_c}$ according to Eq.~\ref{eq:y_init}
		\State $\mb{x^0} \leftarrow H_s\left( \frac{1}{N}\sum_{i=1}^{N}y^{init}_{c,T,i}a_{T,i}\right)$
	\end{algorithmic}
\end{algorithm}
 \begin{figure}[t]
	\begin{subfigure}[b]{0.63\linewidth}
		\centering
		\resizebox{\linewidth}{!}{
			\begin{tikzpicture}[scale=1, every node/.style={scale=1.5,font=\normalsize}]
			\def\normaltwo{\x,{3*1/exp(((\x)^2)/2)}}
			\def\y{4.4}
			
			%\fill [fill=orange!60] (2.6,0) -- plot[domain=0:4.4] (\normaltwo) -- ({\y},0) -- cycle;
			
			% Draw and label normal distribution function
			\draw[color=blue,domain=-4.25:4.25,thick, samples=100] plot (\normaltwo) node[right] {};
			\draw[<->] (-4.25,0) -- (4.25,0);
			\draw (4.25,0.1) node[above left] {$\mathbf{y_c}$};
			\draw[<->] (0,-0.8) -- (0,4);
			\fill [fill=green!60] (0,0) -- plot[domain=-4.25:0] (\normaltwo) -- (0,0) -- cycle;
			\fill [fill=orange!60] (0,0) -- plot[domain=0:4.25] (\normaltwo) -- (0,0) -- cycle;
			\draw (-3,-0.5) node(below) {$-\rho$};
			\draw (-4,-0.5) node(below) {$-R$};
			\draw (3,-0.5) node(below) {$\rho$};
			\draw (4,-0.5) node(below) {$R$};
			\draw [->] (2,2.5)--(4,2.5);
			\draw (2.8,2.8) node(above) {$\mod(\cdot)$};
			\foreach \x in {-3,-4,3,4}
			{        
				\coordinate (A\x) at ($(0,0)+(\x*1cm,0)$) {};
				\draw ($(A\x)+(0,5pt)$) -- ($(A\x)-(0,5pt)$);
				
			}
			%		\draw (-1.5,-0.5) node(below) {$p_i = 1$};
			%		\draw (0,-1.5) node(right) {$f(t) = \mod(t,R)$};
			%		\draw[scale=0.5,domain=-7:0,smooth,variable=\x,blue, ultra thick] plot ({\x},{\x+4});
			%		\draw[scale=0.5,domain=0:7,smooth,variable=\x,blue, ultra thick]  plot ({\x},{\x});
			\end{tikzpicture}}
		\caption{\small{}}
	\end{subfigure}
	\begin{subfigure}[b]{0.36\linewidth}
		\centering
		\resizebox{\linewidth}{!}{
		\begin{tikzpicture}[scale=1, every node/.style={scale=1.4, font=\normalsize}]
		\def\normaltwo{\x,{3*1/exp(((\x)^2)/2)}}
		\def\normalone{\x,{3*1/exp(((\x-4)^2)/2)}}
		\def\normalsum{\x,{3*1/exp(((\x-4)^2)/2)+3*1/exp(((\x)^2)/2)}}
		\def\y{3}
		\def\fy{3*1/exp(((\y-4)^2)/2)}
		\fill [fill=orange!60] (0,0) -- plot[domain=0:1] (\normaltwo) -- (1,0) -- cycle;
		\fill [fill=green!60] (3,0) -- plot[domain=3:4] (\normalone) -- (4,0) -- cycle;
		\fill [fill=gray!30] (1,0) -- plot[domain=1:3] (\normalsum) -- (3,0) -- cycle;
		% Draw and label normal distribution function
		\draw[color=blue,domain=-0:3,dashed,thick] plot (\normaltwo) node[right] {};
		\draw[color=blue,domain=1:4,dashed,thick] plot (\normalone) node[right] {};
		\draw[color=blue,domain=-0:4,thick] plot (\normalsum) node[right] {};
		\draw[<->] (-0.5,0) -- (4.5,0);
		\draw (4.25,0) node[above] {$\mathbf{y}$};
		\draw[<->] (0,-0.8) -- (0,4);
		\draw[dashed] ({\y},{\fy}) -- ({\y},0);
		\draw[dashed] ({4},{3}) -- ({4},0);
		\draw[dashed] ({1},{\fy}) -- ({1},0);
		%\draw (-3,-0.5) node(below) {$-\rho$};
		%\draw (-4,-0.5) node(below) {$-R$};
		\draw (3,-0.5) node(below) {$\rho$};
		\draw (4,-0.5) node(below) {$R$};
		\draw (1,-0.5) node(below) {$R-\rho$};
		%\draw (0.5,1) node(below) {$p_i=0$};
		%\draw (3.5,1) node(below) {$p_i=1$};
		%\draw (2,0.5) node(below) {$p_i=0$};
		
		 \draw[<-](0.63,2)--(0.63,3.5); %node(above right) {$~~p_i=0$};
		 \draw[<-](3.5,2)--(3.5,3.5); %node(above) {$~~p_i=1$};
		 \draw[<-](2,0.6)--(2,1.5); %node(above) {$~~p_i=0$};
		 \draw (0.63,3.6) node(above) {$p_i=0$};
		 \draw (3.5,3.6) node(above) {$p_i=1$};
		 \draw (2,1.6) node(above) {$p_i=0$};
		\foreach \x in {1,3,4}
		{        
			\coordinate (A\x) at ($(0,0)+(\x*1cm,0)$) {};
			\draw ($(A\x)+(0,5pt)$) -- ($(A\x)-(0,5pt)$);
			
		}
		%		\draw (-1.5,-0.5) node(below) {$p_i = 1$};
		%		\draw (0,-1.5) node(right) {$f(t) = \mod(t,R)$};
		%		\draw[scale=0.5,domain=-7:0,smooth,variable=\x,blue, ultra thick] plot ({\x},{\x+4});
		%		\draw[scale=0.5,domain=0:7,smooth,variable=\x,blue, ultra thick]  plot ({\x},{\x});
		\end{tikzpicture}}	
		\caption{\small{}}
	\end{subfigure}
	\caption{\emph{Density plots of (a) $\mathbf{y_c=Ax^*}$; (b) $\mb{y}=\mod(\mathbf{Ax^*})$.}}
	\label{fig:densityplot}
\end{figure}
We define set $T$ as the set of measurements for which we can identify the bin-index correctly. We introduce $N=: \card(T)$.
%\begin{align*}
%N & =  m - \card(\textnormal{region of uncertainity}).
%\end{align*}
Value of $N$ largely depends on the difference between $\rho$ and $R$.
Once we identify the correct bin-index for part of the measurements, we can re-calculate corrected measurements as,
\begin{align}
\mathbf{y^{init}_{c} = y + p^{init}R}.
\label{eq:y_init}
\end{align}
We use these corrected measurements $\mathbf{y^{init}_{c}}$ to calculate the initial estimate $\mathbf{{x}^0}$ with first order unbiased estimator.% using the estimation step described in the upcoming section.
\begin{equation}
\mb{x^0} = H_s\left( \frac{1}{N}\sum_{i=1}^{N}y^{init}_{c,T,i}a_{T,i}\right).
\label{eq:init}
\end{equation}
where $H_s$ denotes the hard thresholding operator that keeps the $s$ largest absolute entries of a vector and sets the other entries to zero.
\subsection{Alternating Minimization}
\label{sec:altmin}
\begin{algorithm}[t]
	\caption{\textsc{MoRAM-descent}}
	\label{alg:MoRAM}
	\begin{algorithmic}
		\State\textbf{Inputs:} $\mathbf{y}$, $\mathbf{A}$, $s$, $R$
		\State\textbf{Output:}  $\widehat{x}$
		\State $m,n \leftarrow \mathrm{size}(\mathbf{A})$ 
		\State \textbf{Initialization}
		\State $\mathbf{x^0} \leftarrow \textrm{MoRAM-initialization}(\mathbf{y, A})$ 
		\State \textbf{Alternating Minimization}
		\For {$l =0:L$}
		\State $\mathbf{{p}^{t}} \leftarrow \frac{\mathbf{1}-\sgn(\langle \mathbf{A} \cdot \mathbf{x^t} \rangle)}{2}$
		\State $\mathbf{y^t_c} \leftarrow \mathbf{y} - \mathbf{p^t}R$
		\State $\mathbf{{x}^{t+1}}\leftarrow \small{JP(\frac{1}{\sqrt{m}}\begin{bmatrix} \mathbf{A} & \mathbf{I} \end{bmatrix},\frac{1}{\sqrt{m}}\mathbf{y^t_c},[\mathbf{x^t~~p^t}]^\t)}$.
		\EndFor
	\end{algorithmic}
\end{algorithm}

In the descent step, starting with $\mathbf{{x}^0}$, we calculate the estimates of $\mathbf{p}$ and $\mathbf{x}$ in alternating fashion to converge to the original signal $\mathbf{x^*}$. At each iteration of our Alternating Minimization, we use the current estimate of the signal ${\mathbf{x^t}}$ to get the value of the bin-index vector $\mathbf{{p}^t}$ as following:
\begin{equation}
\mathbf{{p}^{t}} = \frac{\mathbf{1}-\sgn(\langle \mathbf{A} \cdot \mathbf{x^t} \rangle)}{2}.
\label{step1}
\end{equation}

Given $\mathbf{x^0}$ is close to $\mathbf{x^*}$, $\mathbf{p^0}$ would also be close to $\mathbf{p^*}$. Ideal way is to calculate the correct compressed measurements $\mathbf{y^t_c}$ using $\mathbf{p^t}$, and use $\mathbf{y^t_c}$ with any popular compressive recovery algorithms such as CoSaMP or basis pursuit to calculate the next estimate $\mathbf{{x}_{t+1}}$. 

However, even the small error $\mathbf{d^t} = \mathbf{p^t - p^*}$ would reflect heavily in the calculation of $\mathbf{x^t}$, as each incorrect bin-index would add a noise of the magnitude $R$ in $\mathbf{y^t_c}$. The typical sparse recovery algorithms are not robust enough to cope up with such gross errors in $\mathbf{y^t_c}$ \cite{Laska2009}. To tackle this issue, we employ the justice pursuit based formulation which is specifically robust towards grossly corrupted measurements. We consider the fact that the nature of error $\mathbf{d^t}$ is sparse with sparsity $s_{dt}=\norm{\mb{d^t}}_0$; and each erroneous element of $\mathbf{p}$ adds a noise of the magnitude $R$ in $\mathbf{y^t_c}$. Thus we solve the augmented optimization problem:
$$
\mathbf{{x}^{t+1}}=\argmin_{[\mathbf{x~d}]^\t \in \mathcal{M}_{s+s_{dt}}}\norm{\begin{bmatrix} \mathbf{A} & \mathbf{I} \end{bmatrix} \begin{bmatrix} \mathbf{x} \\ \mathbf{d} \end{bmatrix} - \mathbf{y^t_c}}_2^2, %~~\mathrm{s.to}~~x^* \in \mathcal{M}_s,
$$
%\begin{equation}
%= \cosamp(\frac{1}{\sqrt{m}}\begin{bmatrix} \mathbf{A} & \mathbf{I} \end{bmatrix},\frac{1}{\sqrt{m}}\mathbf{y_c^t},s+s_p,[\mathbf{x^t~~p^t}]^\t).
%\label{eq:robcosamp}
%\end{equation}

However, the sparsity of $\mathbf{d^t}$ is unknown, suggesting that the sparse recovery algorithms taking sparsity as a parameter cannot be used here. Thus, as we employ basis pursuit, which doesn't rely on sparsity. The robust formulation of basis pursuit is referred as Justice Pursuit (JP) \cite{Laska2009} in the literature, specified in Eq.~\ref{eq:jp}.
\begin{equation}
\mathbf{{x^{t+1}}} = JP(\frac{1}{\sqrt{m}}\begin{bmatrix} \mathbf{A} & \mathbf{I} \end{bmatrix},\frac{1}{\sqrt{m}}\mathbf{y^t_c},[\mathbf{x^t~~p^t}]^\t).
\label{eq:jp}
\end{equation}
We repeat the steps of bin-index calculation (as in Eq.~\ref{step1}) and sparse recovery (Eq.~\ref{eq:jp}) alternatively for $L$ iterations.% Our algorithm is able to achieve perfect convergence, as supported by the results in the experiments section.

\subsection{Theoretical results}
We now provide theoretical guarantees and their proof sketches for our proposed algorithm. For brevity, here we omit the proofs, which are discussed in detail in an extended version [CITE]. % of our paper available on author's website \footnote{\url{http://virajshah.me/},\url{http://home.engineering.iastate.edu/~chinmay/}}.
\begin{theorem}[Guarantee for Initialization]
	Let $\delta \in (0,1)$ and $t \geq 1$. The initial estimate $\mb{x^0}$ is the output of the algorithm~\ref{alg:RCM}. If the number of measurements satisfy, $m \geq C(\frac{t}{\kappa})^2s\left(1-2\frac{\phi(R-\rho)}{(R-\rho)} \right)^{-1}$, where $\kappa = \frac{\delta}{2}$, $\phi(\cdot)$ represents standard normal pdf; then with probability at least $1 - 2\left(\frac{en}{s}\right)^s\exp\left(-t^2s\right)$ we have,
	$$
	\norm{\mb{x^0} - \mb{x^*}}_2 \leq \delta \norm{\mb{x^*}}_2.
	$$
	\label{th:1}
\end{theorem}
%For the proof of Theorem~\ref{th:1}, we calculate the upper and lower bounds on $N$ in terms of $m,R$ and $\rho$.%, using the fact that the compressive measurements $\mb{y_c}$ follow the standard normal distribution. 
%We further exploit the sparsity of the underlying signal $\mb{x^*}$ to bound the distance between a fixed sparse signal $\mb{x^*}$ and non-thresholded version of $\mb{x^0}$ using results from~\cite{vershynin2010introduction}. We take union bound over all the $s-$sparse vectors to complete the proof.

For simplicity, we limit our analysis of the descent to only one iteration of Alternating Minimization in our algorithm. In fact, as proven in our analysis, theoretically only one iteration of AltMin with JP is required for exact signal recovery. Contrast to that, in practice we observe that our algorithm requires more than one AltMin iterations to converge to the optimum solution.
\begin{theorem}[Guarantee for Descent]
	Given an initialization $\mb{x^0}$ satisfying $\norm{\mb{x^* - x^0}}_2 \leq \delta\norm{\mb{x^*}}_2$, for $0 < \delta < 1, \eta \in [0,1],~\eps > 0$, if we have number of (Gaussian) measurements satisfying $m \geq \frac{2}{\eps^2}\left(s\log{(n)} + 2s\log{\left(\frac{35}{\eps}\right)}+\log{\left(\frac{2}{\eta}\right)}\right)$ and $s \leq \gamma m/\left(\log\left(n/m\right)+1\right)$, then the estimate after the first iteration $\mb{x^1}$ of Algorithm \ref{alg:MoRAM} is exactly equal to the true signal $\mathbf{x^*}$ with probability at least $1-K\exp(-cm)-\eta$, with $K$ and $c$ being numerical constants.
	\label{th:2}
\end{theorem}
%For proving Theorem~\ref{th:2}, we start with the results pertaining to 1-bit compressive sensing from~\cite{Jacques2013} to prove that the corruption in the first set of corrected measurements $\mb{y_c^0}$ can be modeled as sparse vector with sparsity less than or equal to $cm$, with $c$ being a fraction. We claim the guaranteed recovery of the true signal using the result proved in \cite{li2013compressed} for Gaussian random measurement matrix, which states that one can recover a sparse signal exactly by tractable $\mathit{l1}-$minimization (such as basis-pursuit) even if a positive fraction of the measurements are arbitrarily corrupted.