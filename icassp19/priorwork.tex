\section{Prior work}
\label{sec:prior}
\emph{\textbf{Modulo recovery and phase unwrapping:}} \textcolor{red}{In the literature, typically the modulo recovery problem is considered within the setup of Nyquist sampling \textit{i.e.}, bandlimited (smooth) signal assumption, uniform grid sampling in time/spatial domain etc. being the key characteristics. For example, given the modulo samples of a bandlimited function, \cite{Bhandari} provides a stable algorithm for perfect recovery of the signal and also proves the $O(n)$ sample complexity for the perfect recovery. \cite{Cucuringu2017} formulates and solves an QCQP problem with non-convex constraints for denoising the modulo 1 ($R =1$) samples of the unknown function along with providing a least-square based modulo recovery algorithm. However, both these algorithms consider uniform sampling grid with a bandlimited signal, and thus, not suitable for sparsity based sub-Nyquist sampling scheme. For images, the phase unwrapping algorithm proposed in \cite{bioucas2007phase} (and the subsequent Single-shot UHDR in \cite{ICCP15_Zhao}) is specialized to images, and employs graph cuts for recover true image from a single modulo measurement per pixel. However, the fact that it operates in uniform grid sampling regime with the inherent assumption that the input image has very few sharp discontinuities makes it unsuitable for practical compressive sampling. Our main motivation for this paper is the multi-shot UHDR method of \cite{ICCP15_Zhao}, which reconstructs the true image from multiplexed linear measurements obtained in oversampled setting. There, the measurement matrix has to be chosen carefully, and because the method doesn't use any inherent signal structure, it results in high oversampling factor. In our previous work \cite{Shah}, we proposed an algorithm based on \cite{ICCP15_Zhao, soltani2017stable} for signal recovery from quantized modulo measurements, which can also be adapted for sparse measurements, however, similar to \cite{ICCP15_Zhao}, it doesn't leverage sparsity in the modulo recovery process.}

\emph{\textbf{Phase retrieval:}} Approaches to solve phase retrieval problem can be broadly classified into two categories: convex and non-convex. 
Convex approach usually consist of solving a constraint optimization problem after linearizing the problem. PhaseLift algorithm \cite{candes2013phaselift} and its variations \cite{gross2017improved}, \cite{candes2015phasediff} come under this category. Typical non-convex approaches involve finding a good initialization followed by iterative minimization, \textit{e.g.} Approaches based on Amplitude flow \cite{wang2016sparse,wang2016solving} and Wirtinger flow \cite{candes2015phase, zhang2016reshaped,  chen2015solving, cai2016optimal}.

In recent works, phase retrieval problem for the cases where underlying signal is sparse is of growing interest. Some of the convex approaches for sparse phase retrieval includes \cite{ohlsson2012cprl, li2013sparse,bahmani2015efficient,jaganathan2012recovery}. Similarly, non-convex approaches for sparse phase retrieval includes \cite{netrapalli2013phase, cai2016optimal, wang2016sparse}. Our approach in this paper towards solving the modulo recovery problem is mainly inspired from the non-convex sparse phase retrieval framework advocated in \cite{Jagatap2017}. 


%Given the modulo samples of a bandlimited function, \cite{Bhandari} provides a stable algorithm for perfect recovery of the signal and also proves sufficiency conditions that guarantees the perfect recovery. \cite{Cucuringu2017} formulates and solves an QCQP problem with non-convex constraints for denoising the modulo 1 ($R =1$) samples of the unknown function along with providing a least-square based modulo recovery algorithm. However, both these works relay on the smoothness of the bandlimited function as a prior structure on the signal, and can't be used in our setup.

\begin{table*}[t]
	\centering
	\begin{tabular}{|p{3cm}|p{3cm}|p{2.5cm}|p{3cm}|p{3cm}|}
		\cline{1-5}
		& Unlimited Sampling~\cite{Bhandari}  & OLS Method~\cite{Cucuringu2018} & multishot UHDR~\cite{ICCP15_Zhao}  & MoRAM \\ \cline{1-5}

		Provides sample complexity bounds?  & Yes  & --  & No & Yes \\ \cline{1-5}
		Assumption on structure of signal  & Bandlimited, i.e. smoothness   & Bandlimited, i.e. smoothness &  No assumptions & Sparsity     \\ \cline{1-5}
		Sampling scheme (w.r.to Nyquist criteria)  & oversampled, uniform grid &  --, uniform grid  &  oversampled, (carefully chosen) linear measurements   & under-sampled,
		random linear measurements \\ \cline{1-5}
		Sample complexity  & $ O(n)$    & --    & $Kn$   & $O(slog(n) + Ks)$ \\ \cline{1-5}
		Leverages Sparsity?  & No & No   & No    & Yes \\ \cline{1-5}
		(Theoretical) bound on Dynamic range    & Unbounded   & Unbounded     & Unbounded   & $2R$  \\ \cline{1-5}                 
	\end{tabular}
\end{table*}

%\textcolor{red}{ About comparision between Bhandari et al. and Our approach:: \\
%According to Nyqist-Shannon criterion, a bandlimited signal can be uniquely represented through its samples collected at a sampling frequency twice its bandwidth. Many variations of this theorems are studied, but in most cases the variation arise from the diversity along time dimension, i.e., sparsity vs smoothness or uniform vs non-uniform grid. In Bhandari et al., the theme is varied only based on the amplitude dimension, not on time dimension. It means the setup pertaining to time dimension in Bhandari et al. is same as standard Nyquist sampling set up. This standard setup requires that the signal is smooth in canonical (time/space) basis, and considers uniform grid for sampling. However, the measurement set up along amplitude dimension is modulo. They prove that to recover the signal from the modulo measurements perfectly, the sampling frequency has to be greater than $2e(bandwidth)$.}
%\textcolor{red}{
%Now, just as the Nyquist Shannon sampling and Compressive Sampling is complimentary to each other, ours and Bhandari et al.'s approach seems to be complimentary. Maybe it can be put in this form: In Bhandari et al., the theme is varied only along the amplitude dimension, while in our case, we move away from standard Nyquist sampling by varying the sampling scheme along both the time and amplitude dimension, as we consider both the modulo sampling and sparsity. Also, the method of modulo recovery proposed in Bhandari et al. rely on smoothness, thus cannot be applied in our sparsity based setup. In our case, \textbf{we leverage sparsity for undoing the effect of modulo operation, which is very novel.}}
%
%\textcolor{red}{ About comparision between Tyagi et al. and Our approach:: \\
%The main focus of the work of Tyagi et al. is not the modulo recovery, but the denoising of modulo measurements. However, in the later part of their paper,for completeness, they do provide two different algorithms for modulo recovery once the modulo observations are denoised. Again, the basis of the Tyagi et al. is same as Bhandari at al,  as they also consider a variation along amplitude dimension only (bandlimited signal, smoothness etc. ).Their modulo recovery algorithm rely on smoothness of the function. They do not provide sample complexity bounds as their main focus is on the denoising part. Also, similar to Bhandari et al., the method of modulo recovery proposed in Tyagi et al. rely on smoothness, thus cannot be applied in our sparsity based setup.}
%
%\textcolor{red}{About comparision between Single Shot UHDR, Zhao and Rasker et al.  and Our approach::\\
%	Again, here also the setup is standard Nyquist Shannon, in spatial basis. The key assumption is smoothness of the 'image' in the spatial domain. No sample complexity bounds are provided. similar to Bhandari et al., the method of modulo recovery proposed here also rely on smoothness (in spatial domain, instead of time domain), thus cannot be applied in our sparsity based setup.}
%
%\textcolor{red}{ About comparision between Multi-Shot UHDR, Zhao and Rasker et al.  and Our approach::\\
%	Here, the measurements are taken in spatial domain, but there is no assumption of smoothness. Here the sampling is non-uniform grid (because of the presence of the carefully chosen multipliers implemented through exposure times). In a way it is similar to our setup of non-uniform sampling, but the sparsity  (or any other signal structure) is not leveraged. Thus, the practical sample complexity is very high ( $m$ is at least 2 times the value of $n$). No theoretical analysis provided. This method is compatible to be used with sparsity (as we have used it in our asilomar paper), but the oversampling factor and difficulty in choosing the multipliers are the drawbacks here. Another key difference is: in the modulo recovery process, we are leveraging sparsity directly. Because of sparsity only our recovery is possible. That is better than an approach which doesn't leverage any structure of the signal.}