\section{Prior work}
\label{sec:prior}

\emph{\textbf{Phase retrieval:}} Approaches to solve phase retrieval problem can be broadly classified into two categories: convex and non-convex. 
Convex approach usually consist of solving a constraint optimization problem after linearizing the problem. PhaseLift algorithm \cite{candes2013phaselift} and its variations \cite{gross2017improved}, \cite{candes2015phasediff} come under this category. Typical non-convex approaches involve finding a good initialization followed by iterative minimization, \textit{e.g.} Approaches based on Amplitude flow \cite{wang2016sparse,wang2016solving} and Wirtinger flow \cite{candes2015phase, zhang2016reshaped,  chen2015solving, cai2016optimal}.

In recent works, phase retrieval problem for the cases where underlying signal is sparse is of growing interest. Some of the convex approaches for sparse phase retrieval includes \cite{ohlsson2012cprl, li2013sparse,bahmani2015efficient,jaganathan2012recovery}. Similarly, non-convex approaches for sparse phase retrieval includes \cite{netrapalli2013phase, cai2016optimal, wang2016sparse}. Our approach in this paper towards solving the modulo recovery problem is mainly inspired from the non-convex sparse phase retrieval framework advocated in \cite{Jagatap2017}. 

\emph{\textbf{Modulo recovery:}} The modulo recovery problem is also known in the literature
as phase unwrapping. The algorithm proposed in \cite{bioucas2007phase} is specialized to images, and employs graph cuts for phase unwrapping from a single modulo measurement per pixel. However, the inherent assumption that the input image has very few sharp discontinuities makes it unsuitable for practical situations with textured images. Our main motivation for this paper is the work of \cite{ICCP15_Zhao} on HDR imaging using a modulo camera sensor. It proposes multi-shot UHDR algorithm for image reconstruction using multiple measurements. The more robust version of it is further proposed in \cite{Lang2017}. However, both these methods depend on carefully decided camera exposures and don't include sparsity in their models. In our previous work \cite{Shah}, we proposed an algorithm based on \cite{ICCP15_Zhao, soltani2017stable} for signal recovery from quantized modulo measurements, which can also be adapted for sparse measurements. 

Given the modulo samples of a bandlimited function, \cite{Bhandari} provides a stable algorithm for perfect recovery of the signal and also proves sufficiency conditions that guarantees the perfect recovery. \cite{Cucuringu2017} formulates and solves an QCQP problem with non-convex constraints for recovering the correct samples of the unknown function from its modulo 1 ($R =1$) samples. However, both these methods relay on the smoothness of the bandlimited function as a prior structure on the signal, and can't be used in our setup.
