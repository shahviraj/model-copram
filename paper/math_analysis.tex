\newpage

\section{Mathematical Analysis}

\subsection{Analysis of the Initialization}
We perform the initialization in two steps:
(i) Measurement correction step, (ii) Estimation step.

\subsubsection{Measurement correction step}: In this step, based on the absolute value of the modulo measurements, we identify the measurements for which we can undo the effect of the modulo operation by guessing the correct value of $\mathbf{p}$. We define the set $T$ as set of indices of all such measurements.

We determine the measurements to be corrected based on the method described in section~{\todo{add section}}. 
In this analysis, our goal is to find out the total number of measurements ($N=|T|$), that can be corrected through the \todo{histogram analysis -  change the term}. Only these $N$ measurements would be used in the estimation step.
Each element of $A$ id i.i.d. standard normal, thus, $\mu_{A_{ij}} = 0$ and $\sigma_{A_{ij}}=1$.
$$
y_{c,i} =\langle \mathbf{a_i} \cdot \mathbf{x^*} \rangle = \sum_{j=1}^{n}A_{ij}x^*_{j}.
$$

So, 
$$
y_{c,i} \sim \mathcal{N}\left(\mu= \sum_{j=1}^{n}x^{*}\mu_{A_{ij}}, \sigma =\sum_{j=1}^{n}x^{*2}_{j}\sigma^2_{A_{ij}}\right)
$$
$$
\implies y_{c,i} \sim \mathcal{N}\left(\mu= 0, \sigma =\sum_{j=1}^{n}x^{*2}_{j} = \norm{\mathbf{{x}^*}}^2 = 1 \right)
$$

Here, each element of $\mathbf{y_c}$ follows a zero mean Gaussian distribution with unit variance.

Now to calculate $N$, we consider the following result:
We can calculate $N$ using the fact that the measurements follow the standard normal distribution, and the total number of measurements are $m$.
Number of measurements lying in the interval $[\alpha,\beta]$ = $mP(\alpha \leq y_{c,i}\geq \beta) = m\left(Q(\alpha)- Q(\beta)\right)$; \\
Where, $Q(\cdot)$ is Q-function, defined as, $Q(t) = 1-\Phi(t)$ with $\Phi(\cdot)$ being CDF of standard normal distribution. \\
We also note that $Q(-t) = 1 - Q(t)$. \\
Q-function is not an elementary function. However, it can be bounded by following functions where $\phi(\cdot)$ is standard normal density function:
$$
\left(\frac{t}{1+t^2}\right)\phi(t) < Q(t) < \frac{\phi(t)}{t}.
$$

case (i) $R > \rho$: in this case, \\
$N$ = number of measurements lying in the interval $[-R + \rho,0]$ + number of measurements lying in the interval $[0, R-\rho]$.
\begin{equation}
N = m\left(Q(-R+\rho)- Q(0)\right) + m\left(Q(0)- Q(R-\rho)\right)
~~ = m\left(Q(-(R-\rho))- Q(R-\rho)\right) 
~~ = m\left(1-2Q(R-\rho)\right)
\end{equation}

Using the above bounds,
$$
N \geq m \left(1-2\frac{\phi(R-\rho)}{(R-\rho)} \right)
$$

case (ii) $R < \rho$: in this case, \\
$N$ = number of measurements lying in the interval $[-\rho,-R]$ + number of measurements lying in the interval $[R,\rho]$.
\begin{equation}
N = m\left(Q(-\rho)- Q(-R)\right) + m\left(Q(R)- Q(\rho)\right)
~~ = m\left(1 - Q(\rho)-1 +Q(R) +Q(R) -Q(\rho)\right) 
~~ = 2m\left(Q(R)-Q(\rho)\right)
\end{equation}

Using the above bounds,
$$
N \geq 2m \left(\frac{R}{(1+R^2)} \phi(R) - \frac{\phi(\rho)}{\rho}\right)
$$


\subsubsection{Estimation step}

We calculate the initial estimate using the measurements in set $T$ as following,
$$
\mathbf{{x}^0} = M = \frac{1}{|T|}\sum_{i\in T}y_{i}a_{i}.
$$

We use the versions of $\mb{y}$ and $A$ truncated to the indices belong to set $T$.
$$
\mathbf{{x}^0} = M = \frac{1}{N}\sum_{i=1}^{N}y_{T,i}a_{T,i} =  \frac{1}{N}A^T\mb{y_T}.
$$
Substituing $\mb{y_T}=A_T\mb{x^*}$,
$$
\mathbf{{x}^0} = M = \frac{1}{N}A_T^\t A_T\mb{x^*}.
$$

We note that each row of our Gaussian measurement matrix ($A_{T,i}$) is independent, and also follows the Guassian distribution with zero mean.  
We denote this distribution with Gaussian random vector $Z$ in $\R^n$, and arrange $N$ rows as the independent samples from the distribution; $Z_i := A_{T,i}$. 

Recall that the covariance matrix of $Z$ can be calculated as $\Sigma = \mathbb{E}Z\otimes Z$,
$$
\Sigma= \mathbb{E}Z\otimes Z = \mathbb{E} ZZ^\t = diag(\mathbb{E}z_1^2,\mathbb{E}z_2^2,...,\mathbb{E}z_n^2) = I_n
$$

Now, calculating the sample covariance matrix of $Z$ using the samples $Z_i = A_{T,i}$,
$$
\Sigma_N = \frac{1}{N}\sum_{i=1}^{N}Z_i \otimes Z_i = \frac{1}{N} A_T^\t A_T
$$
Let $\epsilon \in (0,1), t\geq 1$,
We invoke the Corrolary 5.50 of \cite{} \todo{cite vershynin},
\theorem{Let $\epsilon\in (0,1)$ and $t \geq 1$. The initial estimate $\mb{x^0}$ is the output of the algorithm \todo{add algo ref}. If the total number of corrected measurements satisfy, $N \geq C(\frac{t}{\epsilon})^2n$, then  with probability at least $1 - 2\exp\left(-t^2n\right)$ we have,
$$
\norm{\mb{x^0} - \mb{x^*}}_2 \leq \epsilon \norm{\mb{x^*}}_2.
$$
}

\begin{align*}
\norm{\Sigma_N - \Sigma} &\leq \epsilon \\
\implies \norm{\frac{1}{N} A_T^\t A_T - I_n} &\leq \epsilon \\
\implies \sup_{\mb{x} \in \R^n} \frac{\norm{\left(\frac{1}{N} A_T^\t A_T - I_n\right)\mb{x}}_2}{\norm{\mb{x}}_2} &\leq \epsilon \\
\implies \norm{\left(\frac{1}{N} A_T^\t A_T - I_n\right)\mb{x^*}}_2 &\leq \epsilon \norm{\mb{x^*}_2} \\
\implies \norm{\left(\frac{1}{N} A_T^\t A_T\mb{x^*} - \mb{x^*}\right)}_2 &\leq \epsilon \norm{\mb{x^*}_2} \\
\implies \norm{\mb{x^0} - \mb{x^*}}_2 &\leq \epsilon \norm{\mb{x^*}}_2 \\
\end{align*}


\subsection{Analysis of the Convergence}
We perform the Alternative Minimization Algorithm as described in~\ref{alg:MoRAM}, starting with $\mb{x^0}$ calculated as above. 
The first step is to obtain the initial estimate for bin-index vector, $\mb{p^0}$ using $\mb{x^0}$.
$$
\mathbf{{p}^{0}} = \frac{\mathbf{1}-\sgn(\langle \mathbf{A} \cdot \mathbf{x^0} \rangle)}{2}.
$$
If we try to undo the effect of modulo operation by adding back $R$ for the affected measurements based on the bin-index vector $\mb{p^0}$, it would introduce the error equaling $R$ corresponding to the incorrect bin-index values in $\mb{p^0}$.
$$
\mathbf{y^0_c} = \langle \mathbf{A}\mathbf{x^{0}} \rangle = \mathbf{y} - \mathbf{p^0}R,
$$

If the initial bin-index vector $\mb{p^0}$ is close to the true bin-index vector $\mb{p^*}$, then the fraction of noisy measurements in $\mb{y^0_c}$ with respect to total number of measurements $m$ is small enough to treat $\mb{y^0_c}$ as sparsely corrupted measurements. Unlike the general cases of noise, in this case the corruption cannot be modeled as a probability distribution. However, if we model this corruption as a sparse noise, then we can employ Justice Pursuit for a guaranteed recovery of the true signal given (i) sparsity of the noise is a fraction of total number of measurements; (ii) sufficiently large number of measurements are available. In the following theorem, we claim the guaranteed recovery of the true signal as the corruption in the first set of corrected measurements $\mb{y_c^0}$ can be modeled as sparse vector with sparsity less than or equal to $Cm$, with $C$ being a fraction. 

\lemma{Let $\mb{A}$ be the matrix generated as $\mb{A} \sim \mathcal{N}^{m \times n}(0,1)$ and $\mb{x^*, x^0} \in \R^n$ are $s-$sparse vectors satisfying $\norm{\mb{x^* - x^0}}_2 \leq \delta_1$. Let $\epsilon \in (0,1)$. Given the number of measurements $m \geq ***$, following is true with probability exceeding $1 - (****)$:
	
	$$
	d_H(\sgn({\mb{Ax^*}}), \sgn({\mb{Ax^0}})) \leq \delta_1 + \epsilon;
	$$
where $d_H$ is Hamming distance between binary vectors defined as:

$$
d_H(\mb{a,b}) =: \frac{1}{n}\sum_{i=1}^{n}a_i \oplus b_i \textnormal{~~for $n-$ dimensional binary vectors~} \mb{a,b}.
$$ 
}
\textit{Proof.} For proof, we first introduce the concept of binary $\epsilon$-stable embedding as appears in \todo{cite jacques}.
\definition[Binary $\epsilon$-Stable Embedding]{A mapping $F: \R^n \rightarrow \mathcal{B}^m$ is a binary $\epsilon$-stable embedding (B$\epsilon$SE) of order $s$ for sparse vectors if:
$$
d_S(\mb{x,y}) - \epsilon \leq d_H(F\mb{(x)},F\mb{(y)}) \leq d_S(\mb{x,y}) + \epsilon;
$$
for all $\mb{x,y} \in S^{n-1}$ with $|supp(x) \cup supp(y)|\leq s$.
}
 
In our case, let's define the mapping $F: \R^n \rightarrow \mathcal{B}^m$ \todo{add definition of Bm} as:
$$
F(\mb{x})=: \sgn(\mb{Ax});
$$
with $\mb{A} \sim \mathcal{N}^{m \times n}(0,1)$.

Given $m > ****$, using Theorem 3 from \todo{cite jacques} we conclude that $F$ is a B$\epsilon$SE for $s$-sparse vectors.
Thus for sparse vectors $\mb{x^*, x^0}$:
$$
d_H(F\mb{(x^*)},F\mb{(x^0)}) \leq d_S(\mb{x^*,x^0}) + \epsilon
$$

$$
\implies d_H(\sgn({\mb{Ax^*}}), \sgn({\mb{Ax^0}})) \leq \delta_1 + \epsilon.
\qed
$$


\todo{conversion from angle to l2 distance}

\theorem{Given an initialization $\mb{x^0}$ satisfying $\norm{\mb{x^* - x^0}}_2 \leq \delta_1\norm{\mb{x^*}}_2$, for $0 < \delta_1 < 1$, if we have number of (Gaussian) measurements $m > ****$ , then the iterates of Algorithm \ref{alg:MoRAM} satisfy:
$$
\norm{\mb{x^{t+1} - x^*}}_2 \leq \rho_0 \norm{\mb{x^t}-\mb{x^*}}_2,
$$
with probability greater than $1-****$, for $0<\rho_0<1$.
}

\textit{Proof.} We compute the corruption in $\mb{y_c^0}$ as,

$$
\eta = \mb{y_c^0 - y_c} = \mathbf{y} - \mathbf{p^0}R - \mathbf{y} + \mathbf{p^*}R = (\mb{p^*-p^0})R = \Delta\mb{p}R.
$$

Expanding further,
\begin{align*}
\eta  & = \frac{\mathbf{1}-\sgn(\langle \mathbf{A} \cdot \mathbf{x^0} \rangle)}{2} -  \frac{\mathbf{1}-\sgn(\langle \mathbf{A} \cdot \mathbf{x^*} \rangle)}{2} \\
& = \frac{\sgn({\mb{Ax^*}})- \sgn({\mb{Ax^0}})}{2} \\
& = \frac{F\mb{(x^*)} - F\mb{(x^0)}}{2} \\
& = d_H(F\mb{(x^*)}, F\mb{(x^0)}) \\
& \leq \delta_1 + \epsilon = C_1m.
\end{align*}

Justice Pursuit formulation is used to recover the signal in th